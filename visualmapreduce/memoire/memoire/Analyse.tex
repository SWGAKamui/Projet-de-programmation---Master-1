\chapter{Analyse des besoins}

\section{Besoins fonctionnels}
\subsection{Besoins essentiels}

\subsubsection{Importation}

\textbf{• Importer un fichier de données (au format {\tt .csv})\\} L'utilisateur choisit parmi sa propre bibliothèque de fichiers un fichier sous format {\tt .csv} qui contient des lignes de données sur lesquelles il veut appliquer son code {\it MapReduce}. L'extension doit être en {\tt .csv} mais en revanche nous ne limitons pas la taille du fichier. Plus de détails sur le fichier est retrouvé dans le chapitre \ref{ch:implementation} Implémentation.\\

\textbf{• Importer un fichier contenant les fonctions de {\it MapReduce}\\}   
 Donner la main à l'utilisateur pour charger son propre code {\it MapReduce} avec les 3 fonctions à implémenter {\tt map()}, {\tt reduce()} et {\tt getPartition()}. Ce sont ces fonctions qui serviront pour le traitement des données en {\tt .csv} fournit. Le fichier doit donc être au format {\tt .js} mais il n'est pas nécessaire d'en fournir un.\\

\subsubsection{Configuration} 

\textbf{• Configurer le cluster de map\\ } Offrir la possibilité à l'utilisateur de fournir le nombre de machines ainsi que le nombre de coeurs par machine. Par contre, il est limité par un maximum de 20 machines à 24 cœurs chacune pour des raisons expliquées plus tard dans ce document. \\

\textbf{• Configurer le nombre de {\it reducer} du programme\\} Il est possible de préciser que le code {\it MapReduce} sera traité avec un nombre de tâches de {\tt reduce()} ({\it ReduceTask}) bien défini. En effet, la simulation peut varier si on a une seule tâche de {\tt reduce()}, dans ce cas toutes les données seront transférées vers un seul {\tt Slot} et le temps de traitement sera plus long. Par contre, utiliser plusieurs {\it reducer} permet une bonne parallélisation selon les cas. Pour ce nombre, il est obligatoire de ne pas dépasser le nombre de slots existants dans la phase de {\tt map()}.\\

\subsubsection{Modification} 

\textbf{• Modifier les fonctions de {\it MapReduce}\\} C'est un besoin étroitement lié à « Importer un fichier contenant les fonctions de {\it MapReduce} ». Quand l'utilisateur joint son fichier {\tt .js}, celui-ci est directement chargé dans une zone de texte totalement modifiable. Ainsi, il peut voir son propre code contenu dans le fichier mais aussi effectuer directement des modifications dessus sans avoir à l'ouvrir ailleurs dans un éditeur de texte. C'est une fonctionnalité très utile pour pouvoir tester son code sur plusieurs cas et voir à chaque fois la simulation obtenue.  \\

\subsubsection{Visualisation}

\textbf{•  Visualiser la simulation de l'exécution de {\it MapReduce}\\} C'est le service principal offert par notre application. Suivant les entrées de l'utilisateur (données {\tt .csv}, code {\it MapReduce} en {\tt .js} et configuration du cluster), un graphique est généré en dessous de la zone de code. Il est affiché alors l'exécution de {\it MapReduce} selon ses spécifications.\\

\textbf{• Consulter les résultats sur console\\} En plus du graphique, l'utilisateur peut notamment voir les résultats des variables de sorties des différentes phases. Ce besoin est intéressant quand il veut consulter les résultats des phases de {\it partitioner} et de {\it shuffle} qui ne sont pas inclus dans la simulation graphique pour ne pas alourdir la lisibilité de ce dernier.\\

\textbf{• Consulter les données exécutées sur un slot\\} Une fois la simulation affichée et le graphique généré, on obtient l'ensemble des machines avec leurs différents cœurs ainsi que les liaisons qui montrent la distribution des données. Pour consulter les données exécutées sur un slot en particulier, l'utilisateur clique sur ce slot. Une zone à droite de la page est alors remplie avec les différentes lignes exécutées sur ce map ou bien {\tt reduce()} (un slot peut contenir soit un {\tt map()} soit un {\tt reduce()}).\\ 
 
\subsubsection{Exportation}

\textbf{• Exporter le résultat de la simulation\\} L'utilisateur peut récupérer les données de sortie de l'exécution de {\it MapReduce} sous forme d'un fichier de données {\tt .csv}. Ce fichier contient toutes les lignes de résultats ayant chacune la structure suivante : {\tt clé; valeur}.


\subsection{Besoin optionnel}

\textbf{• Exporter les fonctions générées en langage Java\\} Ceci permet d'avoir le code Java du processus {\it MapReduce} équivalent à celui en Javascript utilisé pour traiter les données. Ce besoin est facultatif en vue du manque de temps que nous puissions rencontrer.

\subsection{Priorité des besoins fonctionnels}
En plus du découpage besoins essentiels/besoins optionnels, nous pouvons indiquer la priorité des besoins. (Voir tableau ci-dessous)
\begin{center}
\begin{tabularx}{\textwidth}{|X|X|}
  \hline {\bf Besoins} & {\bf Priorité attribuée} \\[4ex]

  \hline Importer un fichier des fonctions & Élevée\\[2ex]
  \hline Configurer les fonctions & Élevée\\[2ex]
  \hline Visualiser la simulation & Élevée\\[2ex]
  \hline Importer un fichier de données & Élevée\\[2ex]
  \hline Configurer le cluster de map & Moyenne\\[2ex]
  \hline Configurer le cluster de reduce & Moyenne\\[2ex]
  \hline Consolter données d'un slot & Moyenne\\[2ex]
  \hline Consulter résultats sur console & Faible\\[2ex]
  \hline Exporter le résultat de la simulation & Faible\\[2ex]
  \hline Exporter les fonctions en java & Faible\\[2ex]
  \hline
\end{tabularx}
\bigskip
Tableau 2.1: Priorités des besoins fonctionnels
\end{center}
\section{Besoins non fonctionnels}

\textbf{Lisibilité du résultat\\} L'utilisateur doit pouvoir lire sans difficulté le résultat du \textit{MapReduce}. Il faut pouvoir zoomer et naviguer dans le graphique réalisé par \textit{FATuM} pour comprendre les liens entre le \textit{mapper} et le \textit{reducer}. L'utilisateur doit pouvoir cliquer sur les slots pour obtenir une liste des données contenues, et ainsi comprendre leurs répartitions.\\

Pour des raisons de visibilité à l'écran et pour simplifier la lecture du graphique, c'est la distribution des données entre {\tt map()} et {\tt reduce()} (avec pour chacun ses entrées et ses sorties) qui est générée. La distribution suit les configurations apportées.

\textbf{Affichage des données dans la console\\} Le résultat des fonctions \texttt{map()} et \texttt{reduce()} doit être affiché dans la console Javascript pour permettre à l'utilisateur d'obtenir le résultat sans avoir à passer par l'interface graphique générée par \textit{FATuM}.


\section{Prototypes d'interface}%%%%%%%%%%%%%%%%%%%%%%%%%%%%%
Nous montrons à travers les illustrations suivantes (voir les figures \ref{fig:acceuil} à \ref{fig:detail}) les prototypes d'interface\footnote{ou WireFrames en anglais.} de notre application telle que nous l'avons imaginé au début du projet. Cette étape a été importante durant la réalisation du projet parce qu'elle a permis d'avoir une idée globale sur les différents composants graphiques (boutons, zones de texte, etc…) ainsi que les interactions possibles entre l'utilisateur et le programme.\\ 
\begin{figure}[H]
  \centering
    \includegraphics[width=0.75\textwidth]{images/interface/page_accueil.png}
    \caption{Page d'accueil}
    \label{fig:acceuil}
\end{figure}

\begin{figure}[H]
  \centering
    \includegraphics[width=0.75\textwidth]{images/interface/page_a_propos.png}
    \caption{A propos}
    \label{propos}
\end{figure}

\begin{figure}[H]
  \centering
    \includegraphics[width=0.75\textwidth]{images/interface/page_interpret1.png}
    \caption{Page d'interprétation}
    \label{fig:interface}
\end{figure}

\begin{figure}[H]
  \centering
    \includegraphics[width=0.75\textwidth]{images/interface/page_interpret2.png}
    \caption{Affichage du résultat}
    \label{fig:resultat}
\end{figure}

L'application étant affichée sur une seule page web, le résultat de la simulation est directement affichée sur la même page où l'utilisateur remplit les données. Le graphique apparaît en dessous de la première section.

\begin{figure}[H]
  \centering
    \includegraphics[width=0.75\textwidth]{images/interface/page_interpret3.png}
    \caption{Détail d'une machine}
    \label{fig:detail}
\end{figure}

Comme illustré dans la figure ci-dessus, l'utilisateur a la possibilité de consulter les détails d'exécution qui concernent une machine particulière. Lorsque l'utilisateur appuie sur l'icône de l'une des machines, l'ensemble des propriétés qui concernent celle-ci s'affichent à l'écran. Au début du projet, ce besoin n'était pas très clair et nous avons cru comprendre qu'il peut aussi voir quelle partie de la section de code est exécutée sur cette machine. Or, nous nous sommes rendu compte plus tard que ce qu'il peut consulter c'est les données traitées sur chaque slot exécutant une des phases {\tt map()} ou {\tt reduce()}.