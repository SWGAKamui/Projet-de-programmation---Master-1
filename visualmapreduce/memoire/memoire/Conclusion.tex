\chapter*{Conclusion}

Avant toutes choses, le projet \textit{VisualMapReduce} a été l'occasion pour nous de se familiariser avec le concept de \textit{MapReduce}, enseigné en Master 2.

Le projet \textit{VisualMapReduce}, de part le faible nombre de contraintes imposées par le client, a été l'occasion de découvrir le langage Javascript, ainsi que l'environnement de développement associé, notamment l'éventail des outils fournis nativement par les navigateurs web Firefox et Chrome.

La réalisation d'un projet à 4 personnes, divisible en deux parties distinctes (rendu visuel avec la librairie FATuM, interpréteur du code MapReduce) nous a permis de nous confronter à la réalité de la gestion de projet. La répartition des tâches (architecture, programmation, tests unitaires, documentation, rédaction du mémoire) et le temps dédié à chacunes d'entre elles.


