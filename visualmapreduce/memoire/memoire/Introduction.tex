\chapter*{Introduction}
Avec l'augmentation\footnote{2.5 trillions d'octets de données par jour\cite{mapReduceTextProcessing}.} du volume de données à traiter, le BigData (Données Massives) est une problématique qui s'est progressivement introduite en entreprise\cite{JoliaFerrierBigData}. {\it MapReduce} est un algorithme qui tente d'apporter une réponse satisfaisante au traitement d'un grand volume de données en un temps raisonnable.\\

{\it MapReduce} exploite le concept de « {\it diviser pour régner} » dont le fondement est de diviser un large problème en petits sous problèmes indépendants les uns des autres dans le but de les traiter en parallèle\cite{mapReduceTextProcessing}. C'est paradigme de programmation servant à traiter et à générer de grands jeux de données avec un algorithme distribué et parallélisé.\\

L'exécution d'un programme {\it MapReduce} se fait sur un cluster\footnote{Une grappe de machines sur un réseau.} contenant généralement un grand nombre de machines, ce qui rend le débogage compliqué. En effet, si une des machines ne parvient pas à exécuter le programme correctement, le retour n'est pas fait à l'utilisateur et l'identification de l'origine du problème est difficile.\\

Le but de {\it VisualMapReduce} est de simuler le fonctionnement d'un programme {\it MapReduce} sur un cluster de machines homogènes, c'est-à-dire qu'elles possèdent toutes les mêmes caractéristiques matérielles, afin de pouvoir le tester avant son application dans le monde réel pour éviter ce genre de problème.\\


Nous présenterons d'abord le détail du projet, puis nous aborderons les points évoqués dans le cahier des besoins. Nous expliquerons les points techniques écrits ainsi que leurs tests respectifs. Enfin nous aborderons les résultats de l'application et la gestion du temps/tâches.