\chapter{Implémentation}
\label{ch:implementation}
\section{Processus d'exécution}
\subsection{Chargement de la page}

\begin{figure}[H]
  \centering
    \includegraphics[width=0.75\textwidth]{diagram/document.pdf}
        \caption{Chargement de la page}
        \label{fig:chargement}
\end{figure}
Quand l'URL est saisi, la page se met à charger sans aucune intervention de l'utilisateur.
La figure \ref{fig:chargement} représente les différents appels de fonctions lors du chargement de la page. En effet, le chargement se fait par l'initialisation de \textit{FATuM} appelé dans le fichier {\tt view.js} qui appelle la bibliothèque {\tt fatum.js}.

\subsection{Lancement de la simulation}
Lorsque le bouton {\it Run} est cliqué, une série d'actions se produit.
D'abord, le cluster est affiché (sans les connections). Ensuite, le traitement de {\it MapReduce} est effectué. Enfin, les connections entre les slots sont affichées.\\
Ces actions sont représentées par trois parties dans les figures \ref{fig:run}, \ref{fig:ProMapRed} et \ref{fig:connection}.
\begin{figure}[H]
  \centering
    \includegraphics[angle=90,height=0.95\textheight]{diagram/diag_seq_init.pdf}
        \caption{Lancement du bouton RUN}
        \label{fig:run}
\end{figure}

\begin{figure}[H]
  \centering
    \includegraphics[height=0.95\textheight]{diagram/diag_seq_mapReduce.pdf}
        \caption{Process de MapReduce}
        \label{fig:ProMapRed}
\end{figure}

Lors de l'affichage du cluster, les données fournies par l'utilisateur doivent respecter certaines conditions imposées par la fonction {\tt Test\_param} (par exemple entrer un nombre supérieur à 1 pour le nombre de PC). Une fois les conditions d'utilisation passées, on appelle la bibliothèque {\it FATuM} pour afficher les différents slots du cluster de {\tt map()} et de {\tt reduce()} ainsi que les numéros des machines (affichage sans les connections entre les slots).

Une fois les éléments du cluster de la simulation affichés, le traitement des données s'effectue. Après chaque traitement, on effectue un lien entre chaque slot du cluster pour indiquer le transfert des données au sein de ce dernier représenté par des flèches provenant de {\it FATuM}.

\begin{figure}[H]
  \centering
    \includegraphics[width=0.75\textwidth]{diagram/print_connection.pdf}
        \caption{Affichage des connections}
        \label{fig:connection}
\end{figure}

\subsubsection{Téléchargement des résultats}
\begin{figure}[H]
  \centering
    \includegraphics[width=0.75\textwidth]{diagram/download.pdf}
        \caption{Lancement du bouton Download}
        \label{fig:DL}
\end{figure}

Une fois la simulation et le traitement des données effectué. L'utilisateur peut récupérer ces données transformer par ses fonctions {\it MapReduce} en les téléchargeant dans un format {\tt .csv} (Figure \ref{fig:DL}).

\section{Partie graphique}


\subsection{Coté interface utilisateur}
\subsubsection{MaterializeCSS}
Pour l'interface utilisateur, nous avons dû utiliser la bibliothèque graphique \href{http://www.materializecss.com/}{{\it MaterializeCSS}} qui fournit un dossier compressé comprenant tous les fichiers de base d'un site web. Ceci permet d'avoir une architecture de fichiers déjà existante et fournir les fichiers CSS nécessaires pour avoir un visuel optimisé et un design attirant.

Ainsi, il suffit d'appeler un élément graphique de HTML, de lui donner la classe correspondante implémentée dans {\it MaterializeCSS} et le visuel est prêt.

%%
\subsubsection{Structure du site}

Le site est en {\it monopage}, c'est à dire qu'il n'y a pas de présence de plusieurs fichiers HTML à charger ou d'un serveur avec une base de données pour les différents élements.

Pour passer d'un contenu à un autre, nous utilisons l'élément HTML de navigation. Ainsi, cela donne l'effet de trois pages: \textbf{Acceuil}, \textbf{Commencer} et \textbf{À Propos}, alors qu'il ne s'agit que d'un seul fichier HTML.\\

Dans la partie \textbf{Commencer}, nous pouvons séparer cette partie en trois sous-catégories. La première permet à l'utilisateur de saisir ces paramètres (les fichiers de données, ses fonctions map/reduce et les paramètres du cluster).
La seconde permet soit d'écrire les fonctions map/reduce directement sur le site, ou de visualiser le code fournis pour le corriger directement.
Enfin la dernière section concerne la simulation avec {\it FATuM} pour le cluster et une colonne à droite de la page pour l'affichage des données contenues dans un slot.

\subsubsection{Les Loaders}
\begin{figure}[H]
  \centering
    \includegraphics[scale=0.5]{images/loader.png}
        \caption{Loader}
        \label{fig:loader}
\end{figure}
Le temps de chargement du site au démarrage ainsi que lors du lancement de la simulation est long. Cette lenteur peut être interprétée par le temps que met \textit{FATuM} pour se charger. C'est pourquoi nous avons visé à rajouter un "Loader" (voir Figure \ref{fig:loader}) au démarrage de la page (pour signifier le chargement de \textit{FATuM}) et un autre pour le temps de calcul de la simulation. Le loader du démarrage fonctionne. Malheureusement, celui de la simulation n'a pas pu être implémenté. En effet, le site se bloque le temps de calcul ce qui empêche l'utilisation du loader.

\newpage
\subsubsection{Les paramètres}
Les {\tt paramètres} entrés par l'utilisateur doivent remplir certaines conditions. En voici une liste exhaustive.
\begin{itemize}
\item Le cluster du Map doit avoir un nombre de machines compris entre 1 et 20 PCs et un nombre de coeurs compris entre 1 et 24. \\ Cette limitation est due à l'inconfort visuel que peut apporter un très grand cluster. En effet, au delà de ces valeurs, la lisibilité n'est plus assurée. Il s'agit là, d'un choix par le groupe suite aux conseils du client.
\item Le nombre de {\tt reduce()} doit être compris entre 1 et le nombre de slots total du cluster de {\tt map()}.
\item Il n'est pas nécessaire d'importer un fichier {\tt .js} pour les fonctions mapReduce. L'utilisateur peut directement coder  dans la "section code". Nous avons aussi fournit un exemple de code qui est fonctionnel et qui implémente l'exemple de WordCount (comme vu dans la figure 1.2 du premier chapitre).
\item Le fichier csv doit être impérativement fourni.
Si l'utilisateur rafraîchit la page, le contenu du fichier disparaît même si le nom du fichier reste, il faut donc la charger de nouveau.
\end{itemize}


\subsection{Simulation graphique(Fatum)}
Comme demandé par le client, nous avons utilisé la bibliothèque graphique \href{http://www.labri.fr/perso/aperrot/fatum/index.html}{{\it FATuM}} développée au \href{http://www.labri.fr/}{{\it LaBRI}}. Cette bibliothèque permet d'afficher la simulation du cluster (voir Figure \ref{fig:sim}) avec différents composants graphiques et ne peut être utilisée pour l'interface utilisateur contrairement à \textit{MaterializeCSS}.
%image fatum simple avec connection.
\begin{figure}[H]
  \centering
    \includegraphics[scale=0.45]{images/graphiqueExemple.png}
        \caption{FATuM - Simulation}
        \label{fig:sim}
\end{figure}

Nous utilisons plusieurs composants de FATuM:
\begin{itemize}
\item Les Marks
\item Les Connections
\item Le zoom
\item Une partie de la gestion d'un "clic souris"\\
\end{itemize}

\begin{figure}[H]
  \centering
    \includegraphics[scale=0.5]{images/marks.png}
        \caption{FATuM : Marks}
        \label{fig:mark}
\end{figure}
Un {\it Mark} (comme dans la figure \ref{fig:mark}) est un élément sous forme de cercle qui représente un slot du cluster. Ils sont séparés entre eux lorsque la limite de slots par PC est atteinte. Ainsi, chaque PC sont séparés graphiquement. Ces éléments graphiques sont dépendants des données fournies par l'utilisateur.\\

Les {\it connections} sont les flèches qui vont d'un Mark vers un autre. Ils représentent le transfert de données entre les slots. On trouve des connections entre l'input de map et map ainsi que des connections entre map et reduce.\\

Le {\it zoom} permet (si l'on pose le pointeur de la souris dans la zone de simulation gérée par \textit{FATuM}) la gestion de la molette de la souris. Ainsi, en cas d'un gros cluster, l'ensemble reste lisible grâce à ce zoom. \\

Enfin, la gestion du {\it clic souris}. Lors d'un clic dans la zone de simulation \textit{FATuM}, les coordonnées récupérées sont celles de la fenêtre. Elles n'ont donc rien à voir avec celles de \textit{FATuM}. La fonction {\tt windowToModel} nous a permis de transformer les coordonnées du clic en coordonnées compréhensibles par \textit{FATuM} pour pouvoir exécuter le traitement suivant la zone de clic.


\subsubsection{Précision sur la fonction "search\_mark"}

\begin{lstlisting}
function search_mark(x, id) {
    var header_data, tmp_header;
    //type of the mark
    switch (x) {
        case indice_fatum_1:
            tmp_header = "Map Input--";
            break;

        case indice_fatum_2:
            tmp_header = "Map Output--";
            break;

        case indice_fatum_3:
            tmp_header = "Reduce Output--";
            break;
        default:
            return false;
    }
    var min = nb_slot + gap;
    var max = min + nb_slot;
    //researh the true id without the gap
    for (var i = 0; i < nb_pc; i++) {
        if (0 <= id && id < nb_slot) {
            header_data = "Slot " + id + ": --"; //exple: Slot 1: --Map Task--
            print_data(x, id, header_data + tmp_header);
            break;
        } else
        if (min <= id) {
            if (id < max) {
                id = id - (gap * (i + 1));
                header_data = "Slot " + id + ": --";
                print_data(x, id, header_data + tmp_header);
                break;
            }
        }
        min = max + gap;
        max = min + nb_slot;
    }
}
\end{lstlisting}
\newpage
Cette fonction nécessite des précisions. 

En effet, pour différencier les blocs de PCs, nous mettons un décalage (variable \textbf{gap}) entre ces blocs. Ce décalage crée des erreurs d'id de mark par la suite. En effet, lorsque l'on clique sur un slot, son id correspond à sa position dans l'axe des y. Mais à cause de ce décalage, l'espacement est également considéré comme un bloc et les id sont décalés. Il serait donc possible, par exemple, de cliquer dans une zone \textit{vide} entre deux machines et ce clic considère que c'est un id valide.\\

D'autre part, cette fonction permet de détecter quel type de donnée on souhaite afficher (map input, map output ou reduce output). Pour cela, on utilise l'axe des X pour se repérer. A titre d'exemple, un clic qui a en x une valeur correspondant à "indice\_fatum\_1" vaut toute la colonne des slots des map inputs.

\subsubsection{La sortie console}
Lors du démarrage de l'application, l'initialisation de \textit{FATuM} s'effectue. Il est donc normal de voir en console (comme dans la figure \ref{fig:console}) des informations concernant la bibliothèque.

\begin{figure}[H]
  \centering
    \includegraphics[scale=0.5]{images/console_fatum.png}
        \caption{Sortie console de l'initialisation de \textit{FATuM}}
        \label{fig:console}
\end{figure}


\newpage
\section{Partie interpréteur}

\subsection{Les classes}

En Javascript, l'implémentation des classes peut se faire de plusieurs manières. Il existe la façon avec le mot clé {\tt class}. Cette manière, bien que plus claire et plus facile à comprendre, n'est pas supportée par tous les navigateurs. Et comme la portabilité est l'un de nos besoins de qualité, nous avons préféré la manière classique de créer des classes en Javascript.\\

La déclaration est par contre un peu différente des autres langages Orientés Objet comme Java ou C++. Nous donnons l'exemple suivant de l'implémentation de la classe {\tt Job}:\\

\begin{lstlisting}

function Job(map, reduce) { 
    this.map = map;
    this.reduce = reduce;
}

Job.prototype.applyMap = function(data) {
	... //Appel a this.map 
}

Job.prototype.applyReduce = function(data) {
    ... //Appel a this.reduce
}

\end{lstlisting}

%En plus de cette classe, nous utilisons 

\subsection{Retranscription du process mapReduce}

Pour les besoins de notre projet, nous n'avons pas basé notre implémentation du code sur les classes mais plutôt sur les fonctions. Pour cela, nous avons centré le code de l'interpréteur dans un seul fichier {\tt classes.js} qui contient, en plus des classes du projet, les fonctions qui interprètent le code {\it MapReduce} pour fournir les sorties (outputs). Les différentes interactions entre les fonctions (appels, affectations etc..) sont schématisées dans la figure \ref{fig:ProMapRed}.\\

On peut voir tout le processus de {\it MapReduce} qui commence par la fonction {\tt split()}. Ensuite, un Job est créé. Ce job s'occupe de lancer les tâches de map et de {\tt reduce()}. Entre les deux tâches, un objet de type {\tt Partitioner} est créé pour appliquer la phase de partition à travers l'appel à la fonction {\tt applyPartitioner}. Ensuite vient la phase de {\it shuffle} qui récupére le résultat du {\it partitioner} et retourne un résultat que {\tt reduce()} va exploiter.\\

Ce sont les fonctions {\tt mapAllData(), shuffleAllData() et reduceAllData()} qui s'occupent d'appeler respectivement les fonctions {\tt applyMap(), shuffle() et applyReduce()} sur tous les slots du cluster.

\subsection{Fonction split}

C'est la première étape à faire dans un process de mapReduce. Cette fonction récupère les données de l'utilisateur sous format textuel et retourne le contenu dans une liste selon le séparateur qu'il a choisi où chaque élément correspond à une ligne de données dont les éléments sont séparés par des espaces\footnote{Exemple: split("je;suis;un;test" , ";") retourne un objet: {je suis un test}.}.\\
Pour implémenter la fonction, nous avons utilisé les {\bf expressions régulières} ci-dessous:

\begin{itemize}
\item L'un des problèmes qui peuvent s'imposer est que le retour à la ligne n'est pas le même sous les différents systèmes d'exploitation (Linux, Windows et macOS). Sous Windows le retour à la ligne est effectué avec le caractère "\escape{n}" tandis que sous macOS cela est fait avec le caractère "\escape{r}". 	
Nous avons donc utilisé une expression régulière dans la fonction {\tt split()} pour que cela marche quelque soit le système d'exploitation sous lequel il a été écrit.\\

\item Dans la même fonction split, on utilise une autre expression régulière {\tt \verb|/\s([^\s]*)$/|} qui s'occupe d'enlever tous les espaces qui ne sont pas nécessaires dans les lignes de données\footnote{Exemple : "{\tt je    ;    contient  ; des;     espaces }"}. L'idée d'utiliser une expression régulière résulte d'un soucis rencontré qui a fait que chaque ligne générée contienne à la fin un espace inutile et cela avait un impact sur les résultats. Une fois nous nous sommes aperçu du soucis, nous l'avons corrigé.
\end{itemize}

\subsection{Fonction eval}
Le premier défi rencontré a été de trouver un moyen de transformer du code sous format textuel en une fonction paramétrable et exécutable qu'on peut directement appeler. Pour ce besoin, nous utilisons la fonction {\tt eval()} qui existe en Javascript. Cette fonction prend en paramètre la forme textuelle du code entré par l'utilisateur. \\

Après l'appel à {\tt eval()} depuis {\tt generateMapReduceFunctions()} (comme indiqué dans la figure \ref{fig:ProMapRed}), nous obtenons les fonctions, prêtes à être appelées directement dans le script. C'est pour cette raison que nous imposons à l'utilisateur de nommer ses fonctions {\tt map()} et {\tt reduce()}.\\

Toutefois, utiliser cette fonction pose quelques problèmes dont nous citons notamment:

\begin{itemize}
\item {\bf Sécurité}: {\tt eval()} ouvre le code pour des attaques par injection\footnote{C'est un type d'attaque qui permet à un attaquant d'injecter des données dans une application pour exécuter du code malveillant.}. Dans notre projet, vu  que l'application est hébergée dans les serveurs du CREMI alors l'environnement est assez sécurisé. Une tierce personne de l'extérieur ne pourra pas accéder au contenu du site.
\item {\bf Lenteur}: {\tt eval()} est plus lente que les méthodes alternatives. En effet, « {\it l'évaluation nécessite de faire appel à l'interpréteur JavaScript alors que de nombreuses structures sont optimisées par les moteurs JavaScript modernes} »\footnote{\url{https://developer.mozilla.org/fr/docs/Web/JavaScript/Reference/Objets\_globaux/eval}}.
\item {\bf Débogage}: peut être plus difficile (pas de numéro de ligne, etc…).
\end{itemize}

\subsection{Fonction hashCode}
Une fonction {\tt hashCode()} retourne une valeur de hachage \footnote{un entier signé 32-bit.} utile lors de la manipulation des instances. \\
Retranscrire le code {\it MapReduce} requiert une implémentation de cette fonction qui sera utilisée lors de la phase de {\it partitioner}. Dans cette phase, si l'utilisateur ne fournit pas une implémentation de la fonction {\tt getPartition()}, la fonction de hashage est appelée par défaut.\\

Une des difficultés rencontrées est que la fonction de hashage, implémenté en langage Java, n'existe pas en Javascript. Nous avons donc redéfinit la fonction de la manière la plus fidèle possible. Pour gagner du temps, nous avons récupéré l'implémentation en Javascript à partir du lien \href{werxltd.com/wp/2010/05/13/javascript-implementation-of-javas-string-hashcode-method/}{suivant\footnote{\url{http://werxltd.com/wp/2010/05/13/javascript-implementation-of-javas-string-hashcode-method/}}}.


\section{Limitations}

\subsection{Difficultés rencontrées}
{\bf FATuM}\\

Etant des débutants en programmation Web, il nous a fallu du temps pour comprendre comment utiliser la bibliothèque {\it FATuM}. C'est grâce aux documentations que nous nous sommes familiarisés avec la bibliothèque même si elles n'étaient pas évidentes à comprendre au début. Une ancienne version de la documentation nous a été fortement utile pour comprendre les fonctions fournies par la bibliothèque (même si certaines fonctions n'existent plus dans l'implémentation de {\it FATuM} la plus récente).\\ 

Dans la nouvelle version de la documentation, il n'y a que les énoncés des fonctions, et c'est avec le temps que nous avons pris l'habitude de voir les exemples d'implémentation de {\it FATuM} dans l'ancienne documentation puis nous référer à la nouvelle documentation pour voir ce qui a changé.\\

{\bf MapReduce}\\
\begin{itemize}
\item L'une des étapes de réalisation du projet qui a demandé le plus d'effort est la compréhension du fonctionnement de {\it MapReduce}. En effet, nous avons senti l'importance de bien comprendre comment l'exécution se fait puisque notre projet revient à retranscrire l'exécution sans un framework pour {\it MapReduce}.\\
Cela a représenté un réel défi de retranscrire la distribution normalement faite par Hadoop.\\

\item Récupérer les fonctions écrites en Javascript par l'utilisateur et les appliquer sur les données.\\

\item Redéfinir la fonction de hashage puisqu'elle est implémenté en langage Java mais n'existe pas en Javascript. 
\end{itemize}

\subsection{Amélioration possible}
{\bf FATuM}\\
Plusieur éléments peuvent être rajoutés à la partie graphique pour améliorer l'utilisation de l'application. Nous citons quelques uns:\\
\begin{itemize}
\item Le scrolling horizontal des données.
\item Un loader pour la simulation.
\item Une optimisation visuelle du cluster.
\end{itemize}

\paragraph{}
Il existe un conflit avec {\it WebGL} qui peut, dans des cas exceptionnels, causer la non-initialisation de {\it FATuM}. Nous ignorons à quoi ce bogue est dû, mais il arrive que le navigateur aie des soucis avec cet élément. Ce problème peut être lié au cache. Mais nous n'avons pas de solution à proposer.

Concernant la partie visualisation, nous avons pensé à rendre cyclique l'affichage des éléments de la simulation(c'est-à-dire: entrée {\tt map()} puis sortie {\tt map()} puis sortie {\tt reduce()} chacun en une colonne). Nous montrons l'affichage tel que nous l'avons imaginé dans la figure \ref{fig:possibilite}.

\begin{figure}[H]
  \centering
    \includegraphics[width=0.5\textwidth]{images/cyclique.jpg}
        \caption{Affichage cyclique de la simulation}
        \label{fig:possibilite}
\end{figure}
Ainsi, un très gros cluster serait plus lisible que la version d'affichage actuelle. Mais dans ce cas, elle peut créer deux soucis: il n'y a plus la linéarité du {\it MapReduce} comme attendu et les connections de {\it FATuM} sont moins lisibles car elles vont traverser les marks qui sont à l'intérieur.\\

{\bf MapReduce}\\

L'une des améliorations possibles est de libérer de l'espace mémoire à chaque fois qu'une tâche a terminé son exécution. Ceci peut donc servir dans le cas d'un très grand fichier de données.\\

Malheureusement, pour des contraintes de temps, nous n'avons pas réussi à implémenter la phase de {\it combiner} ni l'exportation du code {\it MapReduce} en Java qui auraient complété l'implémentation de {\it MapReduce}.\\

Nous pouvons aussi rajouter la gestion d'erreurs :
Ajouter un analyseur syntaxique pour le code Javascript entré et faire un retour à l'utilisateur sur les erreurs potentiellement commises sur son code.

\section{Indentation du code}

L'indentation du code est garantie par l'outil {\tt js-beautify}\footnote{\url{https://github.com/beautify-web/js-beautify}}
qui uniformise la forme du code javascript que nous avons réalisé. Cela permet de traiter automatiquement une partie de la factorisation du code.

L'utilisation de règles de codage permet de faciliter la lecture du code source produit entre les développeurs et de minimiser la complexité du code qui peux mener sur le long terme à l'introduction de bogues dans le logiciel.

\subsection{Exemple du résultat produit par {\tt js-beautify}}
Le code est donné à titre d'exemple uniquement et ne fais pas parti du code source de notre projet.

\begin{figure}[H]
\begin{lstlisting}
Partitioner.prototype.apply = function(foo, bar) {
  var a=[],tmp;
  var z,y,x,w;

  for(var i=0;i<foo.len;i++){
    y=foo[i];
    z=this.get(y.key,y.val,bar);

    w=false;
    x=0;
    if(a.len==0){
      tmp=[];
      tmp.push(y);
      a.push(tmp);}
    else{
      while(!w && x<a.len){
        if(z==this.get(a[x][0].key,a[x][0].val,bar)){
          w=true;
          a[x].push(y);}
        x++;}
        if(!w){
          tmp=[];
          tmp.push(y);
          a.push(tmp);}}}
  return a;}
\end{lstlisting}
\caption{Code avant passage de {\tt js-beautify}}
\end{figure}

\begin{figure}[H]
\begin{lstlisting}
Partitioner.prototype.apply = function(foo, bar) {
    var a = [],
        tmp;
    var z, y, x, w;

    for (var i = 0; i < foo.len; i++) {
        y = foo[i];
        z = this.get(y.key, y.val, bar);

        w = false;
        x = 0;
        if (a.len == 0) {
            tmp = [];
            tmp.push(y);
            a.push(tmp);
        } else {
            while (!w && x < a.len) {
                if (z == this.get(a[x][0].key, a[x][0].val, bar)) {
                    w = true;
                    a[x].push(y);
                }
                x++;
            }
            if (!w) {
                tmp = [];
                tmp.push(y);
                a.push(tmp);
            }
        }
    }
    return a;
}
\end{lstlisting}
\caption{Code après passage de {\tt js-beautify}}
\end{figure}
