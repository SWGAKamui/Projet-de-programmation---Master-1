\documentclass[a4paper,12pt]{article}
\usepackage[utf8]{inputenc}
\usepackage[french]{babel}
\usepackage{url}

\setlength{\textwidth}{16cm}
\setlength{\marginparwidth}{0cm}
\setlength{\oddsidemargin}{0cm}
\setlength{\headheight}{0cm}
\setlength{\topmargin}{0cm}
\setlength{\headsep}{0cm}
\setlength{\textheight}{25cm}
\setlength{\footskip}{0cm}
\setlength{\marginparsep}{0cm}

\bibliographystyle{alpha} %format des citations

\title{Projet de Programmation\\VisualMapReduce\\}
\author{\textsc{Al Chahid} Kinda, \textsc{Espiaut} Marc-Alexandre, \textsc{Hentati} Imen, \textsc{Yjjou} Oualid}

\begin{document}
	\maketitle
	\section{Éléments bibliographiques}
	
	\begin{itemize}
	\item Le tutorial MapReduce du wiki du projet Apache \cite{ApacheWikiMRTut}.
	\item La vidéo explicative de IBM sur le MapReduce \cite{IBMMRVideo}.
	\item L'explication du MapReduce sur le wiki du projet Apache \cite{ApacheWikiMRInfo}.
	\item Le tutoriel en Français du MapReduce de Hadoop sur TutorialsPoint \cite{TutorialsPointHadoopMR}.
	\item L'article de l'encyclopédie \textit{bdpedia} sur MapReduce \cite{Bdpedia}.
	\item Le livre de Laurent Jolia-Ferrier sur le BigData et Hadoop \cite{JoliaFerrierBigData}.
	\item Hadoop pour les nuls \cite{DummiesHadoop}.
	\item Le Que sais-je ? dédié au Big Data \cite{QuesaisjeBigData}.
	\end{itemize}
	
	\bibliography{exercice_bibliographie} %ref au fichier "program.bib"
\end{document}
